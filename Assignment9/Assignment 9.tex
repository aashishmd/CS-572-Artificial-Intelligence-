%\section*{Collaborators}
%List all your collaborators.

\documentclass[11pt]{article}

\usepackage[margin=1in]{geometry}
\usepackage{amsmath,amsthm,amssymb}
\usepackage{color}
\usepackage{lipsum} % for filler text
\usepackage{fancyhdr}
\pagestyle{fancy}
\usepackage{enumitem}
\usepackage{graphicx}
\usepackage{array}
\usepackage{tabu}

\newenvironment{problem}[2][Problem]{\begin{trivlist}
\item[\hskip \labelsep {\bfseries #1}\hskip \labelsep {\bfseries #2.}]}{\end{trivlist}}

\fancyhead{} % clear all header fields
\renewcommand{\headrulewidth}{0pt} % no line in header area
\fancyfoot{} % clear all footer fields
\newcommand\tab[1][1cm]{\hspace*{#1}}
\fancyfoot[LE,RO]{\thepage}           % page number in "outer" position of footer line
\fancyfoot[RE,LO]{Aashish Dhakal} %your name in footer line

\begin{document}
%\lipsum[1-20]
\title{CS 572(Assignment 9)} %replace X with the appropriate number
\author{Aashish Dhakal\\ %replace with your name
aashish@iastate.edu\\%replace with username
 }      %if necessary, replace with your course title
\date{}


\maketitle
\section*{}

\textbf{Problem 1:}
Given: \\
$P(w_1) = 1/3$ ----- (1)\\
$P(w_2) = 2/3$ ----- (2)\\
$p(x|w_1) \sim N(\theta_1,1)$ ----- (3)\\
$p(x|w_2) \sim N(\theta_2,1)$ ----- (4)\\
$N(\mu, \sigma^2) = \frac{1}{\sqrt{2\pi\sigma^2}}e^{{-\frac{1}{2}}(\frac{x-\mu}{\sigma})^2}$ ----- (5)\\ \\
Now we have, $D_1= \{ 1, 2, 3\}$ and $D_2 = \{ 3, 7\}$ \\
From (3) and (5), we have\\
$L(x|\theta_1) = \frac{1}{\sqrt{2\pi}}e^{{-\frac{1}{2}}(x-\theta_1)^2}$ ------ (6)\\ \\
Now,
$L(\theta_1) = P(D_1|\theta_1) = P (x = 1 | \theta_1) * P (x = 2 | \theta_1) * P (x = 3 | \theta_1)$\\
Taking log on both sides:\\
$L \; L(\theta_1) = ln\; P (x = 1 |\; \theta_1) * ln\; P (x = 2 |\; \theta_1) * ln\; P (x = 3 |\; \theta_1)$\\ \\
$L \; L(\theta_1) = log \frac{1}{\sqrt{2\pi}}e^{{-\frac{1}{2}}(1-\theta_1)^2} * log \frac{1}{\sqrt{2\pi}}e^{{-\frac{1}{2}}(2-\theta_1)^2} *
log \frac{1}{\sqrt{2\pi}}e^{{-\frac{1}{2}}(3-\theta_1)^2}$ ----- (7)\\ \\
Taking partial derivative of (7) and set it equals to 0.\\
$= (1-\theta_1)\; +\; (2-\theta_1)\; +\; (1-\theta_1)\;$ \\
$\Rightarrow 6 \; = 3\theta_1$\\
$\theta_1 \; = \; 2$ \\ \\
Similarly, for $\theta_2$:\\
$=\; (3-\theta_2)\; +\; (7-\theta_2)\;$ \\
$\Rightarrow 10 \; = 2\theta_2$\\
$\theta_2 \; = \; 5$ \\ \\
Therefore, we rewrite: $p(x|w_1) \sim N(2,1); \; p(x|w_2) \sim N(5,1). $ \\ \\
\textbf{Problem 2}\\
The Bayes decision rule for a data point $x$ assigns class (label) $w_1$ to it, if $p(w_1|x) > p(w_2|x)$,
and assigns label $w_2$ otherwise.
\begin{center}
$p(w_1|x) = \frac{p(x|w_1)p(w_1)}{C}; p(w_2|x)= \frac{p(x|w_2)p(w_2)}{C}$,
\end{center}
Where $C = p(x)$, which can be treated as a normalization constant, and is not relevant for the decision.\\
Let's find a condition at which $p(w_1|x) > p(w_2|x)$:\\
\begin{center}
  $\frac{p(x|w_1)}{3}> \frac {p(x|w_2)2}{3}$\\
  $\frac{1}{\sqrt{2\pi}}e^{-\frac{1(x-2)^2}{2}} > 2 \frac{1}{\sqrt{2\pi}}e^{-\frac{1(x-5)^2}{2}}$\\
  $e^{-\frac{1(x-2)^2}{2}} > e^{-\frac{1(x-5)^2}{2}}$\\
\end{center}
Lets' take a log of each part of the inequality:
\begin{center}
  $(-\frac{1}{2})(x^2-4x+4) > log(2) + (-\frac{1}{2})(x^2-10x+25)$\\
  $(-\frac{1}{2})(6x-21) > log(2)$\\
  $x < \frac{-2log(2) + 21}{6} \approx 3.27$
\end{center}
Therefore, for $x\; <\; \sim \; 3.27$, Bayes decision rule assigns label $w_1$ to $x$, $w_2$ otherwise.
\end{document}
}
