%\section*{Collaborators}
%List all your collaborators.

\documentclass[11pt]{article}

\usepackage[margin=1in]{geometry}
\usepackage{amsmath,amsthm,amssymb}
\usepackage{color}
\usepackage{lipsum} % for filler text
\usepackage{fancyhdr}
\pagestyle{fancy}
\usepackage{enumitem}

\newenvironment{problem}[2][Problem]{\begin{trivlist}
\item[\hskip \labelsep {\bfseries #1}\hskip \labelsep {\bfseries #2.}]}{\end{trivlist}}

\fancyhead{} % clear all header fields
\renewcommand{\headrulewidth}{0pt} % no line in header area
\fancyfoot{} % clear all footer fields
\fancyfoot[LE,RO]{\thepage}           % page number in "outer" position of footer line
\fancyfoot[RE,LO]{Aashish Dhakal} %your name in footer line

\begin{document}
%\lipsum[1-20]
\title{CS 572(Assignment 1)} %replace X with the appropriate number
\author{Aashish Dhakal\\ %replace with your name
aashish@iastate.edu\\%replace with username
 }      %if necessary, replace with your course title
\date{}


\maketitle
\section*{}

\textbf{Problem 3.26}
\begin{enumerate}[label=(\alph*)]
  \item The branching factor indicates the number of neighbors in each state so the branching factor $b$ in state space is 4.
  \item There are $4k$ distict states at depth $k$ (for $k>0$).
  \item The maximum number of nodes expanded by the breadth-first tree search is: $((4^{x+y+1}-1)/3)-1$.\\
  At each level: We have $4(x+y)$ distinct nodes, and each node will have 4 extra nodes that can be expanded. \\
  nodes(d) = $1 + 4 + .....+ 4^d$. So, while expanding from origin till $(x,y)$ we will have gone till $(x+y+1)$ as we have to reach till max
  and then subtract 1 from it.
  \item There are quadriatically many states within the square for up to depth $(x,y)$, so the maximum number of nodes expanded by the breadth-first
   search is: $2(x+y)(x+y+1)$ where $(x,y)$ is the destination vertex. \\
   It expands 4 nodes on the first search and it expands extra 4 distinct nodes on every depth from part: b), we would have expanded all the
   nodes by $(x+y+1)$ levels.\\
   $(1+4+8+.....)$ = $(1+4(1+2+...+d)$ where $d = (x+y)$, and solving it we get\\
   $= 2(x+y)(x+y+1)$
  \item Yes. This is the Manhattan district metric and equal to the path length.\\
  For eg: Let $X(0, 0)$ be a state and $G(3, 3)$ be the goals state.\\
  Then, by applying the formula: h = $\left|u-x\left| + \left|v-y\left| = \left|3-0\left| + \left|3-0\left| = 6 $
  \item $(x+y)$ nodes are expanded by $A^*$ graph search using Graph search because the heuristic is perfect.\\
  Example can be the same as problem e.
  \item True. Removing links may induce detours, and increase the length of the shortest path and thatn would result $h$ to be less than or equal
  to the path length so we can say that $h$ remains admissable.
  \item False. Addition of some links between non-adjacent states might result in decrement of the shortest path and that may cause
  $h$ to be greater than the path length and $h$ doesn't remain admissable anymore.
\end{enumerate}

\end{document}
}
