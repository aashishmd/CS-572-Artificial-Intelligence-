%\section*{Collaborators}
%List all your collaborators.

\documentclass[11pt]{article}

\usepackage[margin=1in]{geometry}
\usepackage{amsmath,amsthm,amssymb}
\usepackage{color}
\usepackage{lipsum} % for filler text
\usepackage{fancyhdr}
\pagestyle{fancy}
\usepackage{enumitem}

\newenvironment{problem}[2][Problem]{\begin{trivlist}
\item[\hskip \labelsep {\bfseries #1}\hskip \labelsep {\bfseries #2.}]}{\end{trivlist}}

\fancyhead{} % clear all header fields
\renewcommand{\headrulewidth}{0pt} % no line in header area
\fancyfoot{} % clear all footer fields
\newcommand\tab[1][1cm]{\hspace*{#1}}
\fancyfoot[LE,RO]{\thepage}           % page number in "outer" position of footer line
\fancyfoot[RE,LO]{Aashish Dhakal} %your name in footer line

\begin{document}
%\lipsum[1-20]
\title{CS 572(Assignment 4)} %replace X with the appropriate number
\author{Aashish Dhakal\\ %replace with your name
aashish@iastate.edu\\%replace with username
 }      %if necessary, replace with your course title
\date{}


\maketitle
\section*{}

\textbf{Problem 6.4}
\begin{enumerate}[label=(\alph*)]
  \item For rectilinear floor planning,
  assume the large rectangle has length $L$ and breadth $B$. Each rectangle $R_i$ is parameterized by
  four variables, $x, y, l, b$, which defines its co-ordinates position, length and breadth. All smaller rectangles have
  the set of constraints.\\ \\
  \tab \tab \tab \tab $R_{i,x} \geq 0$,\\ \tab \tab \tab \tab $R_{i,x} + R_{i,l} \leq L$ \\
  \tab \tab \tab \tab $R_{i,y} \geq 0$, \\ \tab \tab \tab \tab $R_{i,y} + R_{i,b} \leq B$ \\ \\
  which makes sure that the smaller rectangles lie within the larger rectangles. In addition to this, we can add constraints
  between two rectangles $i$ and $j$, where $ i \neq j$ \\ \\
  \tab \tab \tab $R_{i,x} + R_{i,l} \leq R_{j,x}$  or  $R_{i,x} \geq R_{j,x} + R_{j,l}$ \\
  \tab \tab \tab $R_{i,y} + R_{i,b} \leq R_{j,y}$  or  $R_{i,y} \geq R_{j,y} + R_{j,b}$ \\
  \\
  The domain of each variable is set of 4-tuples, provided above, that are the right size for the corresponding smaller
  rectangle so that it will fit inside the larger rectangle. \\

  \item Class Scheduling Problem:\\
  Here the provided variables consists of Professors(P), Classrooms(C), Subjects (S) and Time Slots(S).\\
  Let's use $P_{ct}$ and $S_{ct}$ to represent a Professor in classroom $c$ on time $t$ and a subject
  being taught in classroom $c$ on time $t$ respectively.\\
  Now we apply constraints in such a way that now two Professors have classes in the same classroom at the same time,
  that means we need to have all three attributes unique for a particular time slot, i.e. Professor, Classroom and Subject.\\ \\
  The domain of each $P_{ct}$ is the set of professors and domain of each $S_{ct}$ is the set of subjects. Denote
  the set of subjects that professor $P$ can teach by $D(t)$\\ \\
  \tab \tab \tab \tab $P_{ct} \neq P_{dt}$, \tab where $d \neq c$\\ \\
  which implies that no professors are assigned to two different classes at the same time. Also, there is a constraint
  between every $P_{ct}$ and $S_{ct}$, denoted by $C_{ij}(t,s)$ such that it ensures that if a Professor $P$ is being assigned
  to $P_{ij}$, then $S_{ct}$ is assigned a value from $D(t)$.\\
\end{enumerate}
\textbf{Problem 6.7}
\begin{enumerate}[label=(\alph*)]
    \item Let's introduce variables, i.e. Color, Nationality, Candy, Drink and Pet, to represent
    it as a CSP. It will result in having 25 variables. We will represent the house as numbers and keep it constant
    as rows. With this representation we will be able to represent all constraints in a simple way and each
    of the 25 variables will be able to take values from appropiate domains.\\
    Besides that we can also create another representation such that we have five variable for each house, with
    one having domain as color, one with nationality, one with candy, one with drink and one with pet. This representation
    will not be as simple as the previous one.
\end{enumerate}

\end{document}
}
